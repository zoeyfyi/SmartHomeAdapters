\documentclass[onecolumn]{IEEEtran}

% Basic required LaTeX packages
\usepackage[english]{babel} 
\usepackage[utf8]{inputenc} 
\usepackage[T1]{fontenc}

% Packages which allow format customisation
\usepackage{titling}
\usepackage{cite}
\usepackage{caption}
\usepackage{textcomp}
\usepackage{xcolor}
\usepackage{etoolbox} 
\patchcmd{\section}{\centering}{}{}{} % uses the etoolbox package to left align section headings

% Packages which help display things like maths and images correctly
\usepackage{amsmath,amssymb,amsfonts}
\usepackage{algorithmic}
\usepackage{graphicx}
\usepackage{hyperref}

\usepackage{subcaption}
\usepackage{tabularx}
\newcolumntype{L}{>{\raggedright\arraybackslash}X}

% Colored text for goals
\newcommand{\easy}{{\color{green} \textbf{Easy}}}
\newcommand{\medium}{{\color{orange} \textbf{Medium}}}
\newcommand{\hard}{{\color{red} \textbf{Hard}}}

% All your code should be in between the \begin{document} and \end{document} tags otherwise your compiler willl throw an error
\begin{document}


\title{Smart Home Adapters - Project Plan}
\author{Ben Sheffield, Theo Olausson, Gwion Ap Rheinallt, Luke Drennan, \\Guanghui Han, Sameer Karim, David Wang, Spencer Mccann}
\date{25/01/2019} % Leave blank to omit date. Comment out to include today's date or you can add a specific date

\maketitle

\section{Concept}
Smart home adapters transform the user’s existing tech into smart devices. The adapters will come with a selection of grips which can be swapped out to perform the different actions. Users will select the type of devices our product will be interacting with, such as a dial, switch, or button, and then our adapter will switch, twist or press to perform the desired operation. Going further we plan to integrate our device with other smart home solutions such as Alexa and IFTTT (If this then that) to provide a cohesive smart home experience.

For this project we are focusing on 3 use cases:

\begin{itemize}
    \item \textbf{Smart thermostat:} Current devices such as the Hive and Nest replace your existing thermostat, thus requiring landlord’s permission. Our device would go over your existing thermostat, reaching an audience not served by the competition.
    \item \textbf{Switches:} Existing products such as Philips Hue bulbs allow you to turn on/off and change the colour of your lights however they only support the most common bulb types. Our product goes over the existing light switch and is therefore compatible with all bulb types. This is not just useful for lights, but any switch. Often there are switches hidden behind appliances in hard to reach places, for example, a fridge in the garage. Our adapters connect with the user’s devices wirelessly, alleviating the need to pull out the fridge just to turn it off when not in use.
    \item \textbf{Bolt locks:} A linear sliding grip would allow a user to control a garden gate or other bolt lock mechanism. This could be useful for unlocking the gate for the delivery driver when you’re away from home. Again, existing solutions are not applicable to people living in rented homes.
\end{itemize}

\begin{figure}
    \centering
    \includegraphics[width=1.0\linewidth]{architecture.png}
    \caption{Architecture of our system.}
\end{figure}

\subsection{Robot}

The robot is split into 2 parts: the controller and the grips. The controller includes an Arduino, an array of sensors, servo and a battery pack. The grips attach to the controller, when the servo turns the grips manipulate whatever device they are interacting with. The robots themselves attach to the devices with self-adhesive, thus they can easily be removed.

\begin{figure}
    \begin{subfigure}{.5\textwidth}
      \centering
      \includegraphics[width=.8\linewidth]{light-grip-1.jpg}
    \end{subfigure}
    \begin{subfigure}{.5\textwidth}
      \centering
      \includegraphics[width=.8\linewidth]{light-grip-2.jpg}
    \end{subfigure}
    \caption{First iteration of a light switch grip.}
    \label{fig:fig}
\end{figure}

\subsection{Web server}

The web server is actually a collection of many smaller web servers since we will be following the microservice architecture. The publicly accessible servers are:

\begin{itemize}
    \item The android app receives data and sends commands via our client API served via the client server. 
    \item Alexa sends alexa requests to our alexa server.
    \item The robot is connected via a web socket to the root server
\end{itemize}

The internal services correspond to each of the use cases. They keep track of the state the device is in, what actions can be performed, and what servo commands need to be sent. This architecture would allow us to point to user made use cases in the future.

\subsection{Android app}

The android app is the primary user interface into our system. It allows the user to register, calibrate and control their robots.

\begin{figure}
    \begin{subfigure}{.33\textwidth}
      \centering
      \includegraphics[width=.9\linewidth]{app-1.png}
    \end{subfigure}
    \begin{subfigure}{.33\textwidth}
        \centering
        \includegraphics[width=.9\linewidth]{app-2.png}
    \end{subfigure}
    \begin{subfigure}{.33\textwidth}
        \centering
        \includegraphics[width=.9\linewidth]{app-3.png}
    \end{subfigure}
    \caption{Quick mockup of the Android app.}
    \label{fig:fig}
\end{figure}

\section{Goals}

Our goals are what we want to achieve in each sprint (see organisational structure -> agile). We aim at having our sprints be approximately two weeks in length, which matches the time between the demos. In each demo we aim to present a few aspects of our system:

\subsection{Demo 1}

\subsubsection{Demo 1.1 - Register and login via the app}

\begin{center}
    \begin{tabularx}{0.8\linewidth}{|c|L|c|}
        \hline
        \textbf{Component} & \textbf{Task} & \textbf{Difficulty} \\
        \hline
        Web server & Register user with email and password & \easy \\
        \hline
        Web server & Authenticate the user and return token & \medium \\
        \hline
        Web server & Login/register UI & \easy \\
        \hline
        Web server & Send auth token in REST requests & \medium \\
        \hline
    \end{tabularx}
\end{center}

\subsubsection{Demo 1.2 - Switch the light on/off from the Android app}

\begin{center}
    \begin{tabularx}{0.8\linewidth}{|c|L|c|}
        \hline
        \textbf{Component} & \textbf{Task} & \textbf{Difficulty} \\
        \hline
        Web server & Store the state of the switch & \easy \\
        \hline
        Web server & Endpoint for turning light on/off & \easy \\
        \hline
        Web server & Endpoint for retrieving a list of robots & \easy \\
        \hline
        Web server & Establish WebSocket with robot & \easy \\
        \hline
        Web server & Send servo commands on light on/off request & \easy \\
        \hline
        Android & UI to display a list of robots & \easy \\
        \hline
        Android & Light switch toggle UI & \medium \\
        \hline
        Android & Send light on/off request & \medium \\
        \hline
        Robot & Establish websocket with server & \easy \\
        \hline
        Robot & Receive servo commands & \medium \\
        \hline
        Robot & Construct the LEGO switch gripper & \medium \\
        \hline
        Robot & Connect the servo to Arduino & \easy \\
        \hline
        Robot & Create temporary client on DICE to forward servo commands (since the Arduino we have doesn’t have a Wifi chip) & \medium \\
        \hline
    \end{tabularx}
\end{center}

\subsection{Demo 2}

\subsubsection{Demo 2.1 - Walkthrough the robot registration process}

\begin{center}
    \begin{tabularx}{0.8\linewidth}{|c|L|c|}
        \hline
        \textbf{Component} & \textbf{Task} & \textbf{Difficulty} \\
        \hline
        Web server & Endpoint to register a robot to the users account & \easy \\
        \hline
        Web server & Only return the list of robots registered to the user's account & \easy \\
        \hline
        Android & UI for registering a new robot & \hard \\
        \hline
        Robot & Assign each robot a hard-coded ID & \easy \\
        \hline
    \end{tabularx}
\end{center}

\subsubsection{Demo 2.2 - Demonstrate robot calibration steps on the app}

\begin{center}
    \begin{tabularx}{0.8\linewidth}{|c|L|c|}
        \hline
        \textbf{Component} & \textbf{Task} & \textbf{Difficulty} \\
        \hline
        Web server & Endpoint to send calibration parameters including a range, min and max values & \easy \\
        \hline
        Web server & Endpoint to receive calibration parameters & \easy \\
        \hline
        Android & Dynamically build UI for calibrating the robot & \hard \\
        \hline
        Android & Send calibration parameters back to the server & \medium \\
        \hline
    \end{tabularx}
\end{center}

\subsubsection{Demo 2.3 - Adjust temperature via the Android app}

\begin{center}
    \begin{tabularx}{0.8\linewidth}{|c|L|c|}
        \hline
        \textbf{Component} & \textbf{Task} & \textbf{Difficulty} \\
        \hline
        Web server & Store current temperature settings & \easy \\
        \hline
        Web server & Endpoint to get the current temperature & \easy \\
        \hline
        Web server & Endpoint to update the current temperature & \easy \\
        \hline
        Android & UI to choose the desired temperature & \easy \\
        \hline
        Android & Send the desired temperature to the server & \medium \\
        \hline
        Robot & Construct the LEGO dial grip & \hard \\
        \hline
    \end{tabularx}
\end{center}

\subsection{Demo 3}

\subsubsection{Demo 3.1 - Physical demonstration of 3D printed version}

\begin{center}
    \begin{tabularx}{0.8\linewidth}{|c|L|c|}
        \hline
        \textbf{Component} & \textbf{Task} & \textbf{Difficulty} \\
        \hline
        Robot & 3D print case for controller & \hard \\
        \hline
        Robot & 3D print light switch grip & \hard \\
        \hline
    \end{tabularx}
\end{center}

\subsubsection{Demo 3.2 - Asking Alexa to turn a light switch on}

\begin{center}
    \begin{tabularx}{0.8\linewidth}{|c|L|c|}
        \hline
        \textbf{Component} & \textbf{Task} & \textbf{Difficulty} \\
        \hline
        Web server & Create Alexa skills schema & \easy \\
        \hline
        Web server & Add Alexa endpoints for receiving Alexa requests & \hard \\
        \hline
    \end{tabularx}
\end{center}

\subsubsection{Demo 3.3 - Present unlocking and locking a bolt lock via the app}

\begin{center}
    \begin{tabularx}{0.8\linewidth}{|c|L|c|}
        \hline
        \textbf{Component} & \textbf{Task} & \textbf{Difficulty} \\
        \hline
        Web server & Endpoint for unlocking/locking bolt lock & \easy \\
        \hline
        Android & UI for unlocking/locking bolt lock & \easy \\
        \hline
        Android & Send request to lock/unlock bolt lock & \medium \\
        \hline
        Robot & Build grip to manipulate bolt lock & \hard \\
        \hline
    \end{tabularx}
\end{center}

\subsection{Demo 4 (Final Demo)}

\subsubsection{Demo 4.1 - Demonstrate push notification on low battery}

\begin{center}
    \begin{tabularx}{0.8\linewidth}{|c|L|c|}
        \hline
        \textbf{Component} & \textbf{Task} & \textbf{Difficulty} \\
        \hline
        Web server & Create a push notification server & \easy \\
        \hline
        Web server & Receive battery status from the robot and send a push notification when robots battery is low & \easy \\
        \hline
        Android & Register for push notifications with the Android OS & \easy \\
        \hline
        Robot & Send battery status over WebSocket & \easy \\
        \hline
    \end{tabularx}
\end{center}

\subsubsection{Demo 4.2 - Demonstrate turning a light switch on when it’s dark}

\begin{center}
    \begin{tabularx}{0.8\linewidth}{|c|L|c|}
        \hline
        \textbf{Component} & \textbf{Task} & \textbf{Difficulty} \\
        \hline
        Web server & Store sensor data sent from the robots & \medium \\
        \hline
        Web server & Endpoint for requesting history of the sensor data & \medium \\
        \hline
        Web server & Create endpoints for adding/deleting/updating triggers & \medium \\
        \hline
        Web server & When a triggers condition is met send the servo command & \easy \\
        \hline
        Android & UI for configuring a trigger & \easy \\
        \hline
    \end{tabularx}
\end{center}

\subsection{Extensions}

If there is time remaining at the end of the project, there are a number of extensions we could add.

\begin{itemize}
    \item \textbf{Fall detection:} Using an accelerometer we could detect if the device falls off the wall and send the user a notification.
    \item \textbf{IOS app:} An IOS app would allow us to integrate with Siri and Apple home app. It would also expand our market.
    \item \textbf{Web app:} Allow for remote control from anywhere without installing an app.
    \item \textbf{Android app (extensions):} Google home \& Bixbe integration would bring voice control to more devices.
    \item \textbf{If this then that:} IFTTT is a popular tool for making smart home devices work together. Integrating with IFTTT will provide a more cohesive smart home experience.
    \item \textbf{Machine learning:} Using sensor data we could automatically discover rules such as:
        \begin{itemize}
            \item Turn the heating on at 7 am
            \item Turn the lights off at 11 pm
            \item Etc
        \end{itemize}
    \item \textbf{Developer API and the marketplace:} We could provide an API so users can use the hardware in new and creative ways. The marketplace would allow developers to list their extensions and users to discover new applications for our device.
    \item \textbf{Product website:} It would be nice to have a product website to enhance our stand at the trade fair. It would include:
        \begin{itemize}
            \item Developer API documentation
            \item Shopping cart and product selection functionality
            \item Product showcase
        \end{itemize}
\end{itemize}

\section{Time planning}

\begin{figure}
    \centering
    \includegraphics[width=0.8\linewidth]{time.png}
    \caption{Breakdown of how many hours each of our team members will spend on various aspects of the project.}
\end{figure}

\subsection{Time breakdown}

\subsubsection{Workshops}

Not everyone is sure what workshops they want to attend, so we are allocating 3 workshops (3 hours) per person.

\subsubsection{Weekly meetings}

Weekly meetings will occur on a Monday from 15:00 to 17:00. The first half of each meeting will also include our mentor.

\subsubsection{Report meetings}

Besides the weekly meetings, we will have three 2 hour meetings for each report.

\subsubsection{Demo days and final presentations}

We will allocate 2 hours per demo, this including time to prepare for the demo and time to reflect on the feedback we get afterwards.

\subsubsection{Trade fair}

The trade fair is from 9:00 till 17:30, but we will also be allocating another 2 hours to ensure our robot and display is ready.

\subsubsection{Development}

The rest of our time will be working on our assigned components (see organisational structure -> components). This includes building, programming, testing, bug fixing and code reviews. There are 28 easy tasks, 13 medium tasks and 7 hard tasks. If we allocate 10 hours for easy tasks, 20 hours for medium tasks and 30 hours for hard tasks we get 750 hours in total out of the 980 man hours we have available. This leaves us with plenty of time in case we have underestimated the complexity of some goals.

\subsection{Skills}

\subsubsection{Web services}

The web server is the largest most complex part of our system, however both Ben and Gwion have produced similar sized web services in the past. Ben recently worked on a prototype event system for the UoE during an internship. Gwion has developed several web services for the UoE during an internship and as part-time work.

\subsubsection{Android}

A large majority of the team have had experience working on mobile apps through ILP, and a couple of team members have experience outside of university: Ben worked on a fully featured Github client with another person, and Luke has some experience with Swift if we decide to port the app to IOS. We are therefore confident that we already have the skills necessary to complete the easier tasks such as UI to display a list of robots (demo 1.2). Goals such as "Register for push notifications with the Android OS" (demo 4.1) will require some research, however we are familur with the Android documentation so we know were to look.

\subsubsection{Hardware}

Luke has worked on FTC robots for four years; while it wasn’t LEGO he is confident his skills will be transferable. Spencer and Sameer have experience with electronics as part of their degrees.

\subsubsection{3D printing}

A couple of team members have done some 3D modelling before. Wang has 3D printed some simple cases over the break.

\subsection{Gantt Chart}

% TODO: insert gantt chart

\end{document}